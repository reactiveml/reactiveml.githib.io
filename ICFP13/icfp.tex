%\documentclass{llncs}
\documentclass[9pt,preprint]{sigplanconf}

%\documentclass[a4paper]{article}

%\usepackage[utf8]{inputenc}
%\usepackage[T1]{fontenc}
%\usepackage[style=alphabetic]{biblatex}
%\usepackage{csquotes}
\usepackage[english]{babel}
\usepackage{url}
\usepackage[usenames,dvipsnames]{xcolor}
\usepackage{amsmath,amsfonts,amssymb,amsthm}
\usepackage{mathpartir}
\usepackage{listings}
%\usepackage{cprotect} % for listings in footnotes
\usepackage{pgf}
\usepackage[colorlinks,
            linkcolor=black,
            citecolor=black,
            urlcolor=black]{hyperref}
%\usepackage{array}
%\usepackage{multirow}
\usepackage[caption=false]{subfig} % sigplanconf complains without caption=false
\usepackage{rml}
%\usepackage{graphicx}

%\renewcommand{\ttdefault}{txtt}

\usepackage{cancel}

\newcommand\mathid[1]{\;\mathtt{#1}\;}
\newcommand\mathidl[1]{\mathtt{#1}\;}
\newcommand\mathidr[1]{\;\mathtt{#1}}

% Syntaxe abstraite - Expressions
\newcommand\elam[2]{\lambda #1. #2}
\newcommand\eapp[2]{#1\; #2}
\newcommand\erec[2]{\mathidl{rec} #1 = #2}
\newcommand\eproc[1]{\mathidl{process} #1}
\newcommand\erun[1]{\mathidl{run} #1}
\newcommand{\epause}{\mathidl{pause}}
\newcommand\eseq[2]{#1 ; #2}
\newcommand\elet[3]{ \mathidl{let} #1 = #2 \mathid{in} #3 }
\newcommand\eletpar[5]{ \mathidl{let} #1 = #2 \mathid{and} #3 = #4 \mathid{in} #5 }
\newcommand\epar[2]{#1 \,||\, #2}
\newcommand\esig[4]
   { \mathidl{signal} #1 \mathid{default} #2 \mathid{gather} #3 \mathid{in} #4 }
\newcommand\esigshort[2]{ \mathidl{signal} #1 \mathid{in} #2 }
\newcommand\epres[3]{ \mathidl{present} #1 \mathid{then} #2 \mathid{else} #3 }
\newcommand\eemit[2]{ \mathidl{emit} #1\; #2 }
\newcommand\eawait[3]{ \mathidl{await} #1(#2) \mathid{in} #3 }
\newcommand\eawaitim[1]{ \mathidl{await} \mathidl{immediate} #1 }
\newcommand\eawaitimone[3]{ \mathidl{await} \mathidl{immediate} \mathidl{one} #1(#2) \mathid{in} #3 }
\newcommand\efordopar[4]{\mathidl{for} #1 = #2 \mathid{to} #3 \mathid{dopar} #4}
\newcommand\eite[3]{\mathidl{if} #1 \mathid{then} #2 \mathid{else} #3}
\newcommand\eloop[1]{\mathidl{loop} #1}
\newcommand\elast[1]{ \mathidl{last} #1}
\newcommand\euntil[4]{\mathidl{do} #1 \mathid{until} #2(#3) \rightarrow #4 }
\newcommand\ewhen[2]{ \mathidl{do} #1 \mathid{when} #2 }
\newcommand\evoid{\mathtt{()}}

% Semantique grands pas
%\newcommand\bigstep[6]{ #1 \vdash #2 \xrightarrow[#6]{#4, #5} #3 }
\newcommand\bigstep[6]{#2 \xrightarrow[#6]{#4, #5} #3 }

% Types
\newcommand\rt{\tau}
\newcommand\rs{\sigma}
\newcommand\rk{\kappa}

\newcommand\rtvar{\alpha}
\newcommand\rtdot{.}
\newcommand\rtprod[2]{#1 \times #2}
\newcommand\rtarrow[2]{#1 \rightarrow #2}
\newcommand\rtproc[2] { #1 \,\mathtt{process}[#2] }
\newcommand\rtsig[2]{ (#1, #2) \,\mathtt{event} }
\newcommand\rtsigck[3]{ (#1, #2) \,\mathtt{sig}\{#3\} }
\newcommand\unit{\mathtt{unit}}

% Effets
\newcommand\effvar{\phi}
\newcommand\effpar[2]{ #1 \mid\mid #2 }
\newcommand\effcon[2]{#1 ; #2}
\newcommand\effor[2]{#1 + #2}
\newcommand\effrec[2]{\mu #1.\, #2}
\newcommand\effplus{\bullet}
\newcommand\effmin{\textsf{0}}
\newcommand\efftop{\top}
\newcommand\effrun[1]{\mathidl{run} #1}
\newcommand\effinf[1]{#1^\infty}

\newcommand\effeq[2]{#1 \equiv #2}
\newcommand\wf[2]{#1 \vdash #2}
\newcommand\noinst[1]{#1\downarrow^{\bullet}}

% Typage
\newcommand\tyenv{\Gamma}
\newcommand\tyconcat{,}
\newcommand\tyju[4]{ #1 \vdash #2 : #3 \mid #4 }
\newcommand\tyjus[4]{ #1 \vdash_{S} #2 : #3 \mid #4 }

% Unification
\newcommand\mW{\mathcal{W}}
\newcommand\mM{\mathcal{M}}
\newcommand\mU{\mathcal{U}}
\newcommand\mlet{\mathtt{let}\,}
\newcommand\mIn{\mathtt{in}\,}
\newcommand\mand{\mathtt{and}\,}
\newcommand\fresh[1]{#1 \text{ fresh}}
\newcommand\fail{\mathtt{fail}}


% Preuve de typage
\newcommand\rkfst[1]{\mathtt{fst}(#1)}
\newcommand\tyeq[2]{#1 \equiv #2}



\bibliographystyle{plain}

\newcommand{\todo}[1]{ {\color{red} #1}}

%\newcommand{\deq}{\stackrel{\text{\tiny def}}{=}}
\newcommand{\deq}{\triangleq}
\newcommand{\sdeq}{::=}
\newcommand\rulename[1]{\text{(\textsc{#1})\;}
}

\newcommand\topck{\texttt{topck}}

\lstset{columns=fixed, basicstyle=\ttfamily, mathescape=true,
	showstringspaces=false}
\lstset{
basicstyle=\small\ttfamily,
keywordstyle= \color{Blue}\bfseries,
commentstyle=\color{Brown},
stringstyle=\color{red}}
\lstset{language={[par]rml},
aboveskip=0.3em}

\lstMakeShortInline{`}

\newcommand{\kword}[1]{\ensuremath{\mathtt{#1}}}

\newcommand\reactivec{\textsc{ReactiveC}}

\begin{document}

\abovedisplayshortskip=0.3em
\belowdisplayshortskip=0.3em
\abovedisplayskip=0.3em
\belowdisplayskip=0.3em

\newtheorem{theorem}{Theorem}[section]
\newtheorem{property}{Property}
\newtheorem{definition}{Definition}
\newtheorem{lemma}{Lemma}

\title{Reactivity of Cooperative Systems}
%\thanks{This work is supported by the INRIA Action d'envergure Synchronics.}
\subtitle{Application to \rml}

\authorinfo{}{}{}
%\authorinfo{ %\vspace*{-.5cm}
%           Louis Mandel \and C\'edric Pasteur \and
%            Marc Pouzet}
%{DI, \'Ecole normale sup\'erieure, 45 rue d'Ulm, 75230 Paris, France}
%{Firstname.Name@ens.fr}

\maketitle

\begin{abstract}
Cooperative scheduling enables efficient sequential implementation of concurrency. But it is up to the programmer to cooperate to ensure reactivity of the program. We present an analysis that checks the reactivity of programs by abstracting programs into so-called behaviors using a type-and-effect system. Our work is applied to the functional synchronous language \rml. We prove the soundness of our analysis with respect to the big-step semantics: a well-typed program is reactive.
\end{abstract}

\category{D.3.2}{Language Classifications}{Applicative (functional) languages; Concurrent, distributed, and parallel languages}
\category{D.3.3}{Language Constructs and Features}{Concurrent programming structures}
\category{D.3.4}{Processors}{Compilers}

\terms
Languages, Theory

\keywords
Synchronous languages; Functional languages;  Semantics; Type systems

\section{Introduction}
\label{sec:introduction}

Contributions of the paper:
\begin{itemize}
\item A reactivity analysis implemented as a type-and-effect system
\item A new approach to subeffecting using row polymorphism
\item The proof of soundness of the analysis
\end{itemize}

\section{Problem statement}

\subsection{A first example}

A simple example to introduce \rml{} and the problem of reactivity

\subsection{Intuitions and limitations}

\begin{itemize}
\item No value analysis: false positives with some recursive processes
\item No analysis of the presence of signals
\item Reactivity does not equal real-time
\end{itemize}

\section{Big-step Semantics}

In this section, we give an overview of the big-step semantics of \rml{}, also called the behavioral semantics. The interested reader can refer to~\cite{Mandel:2005} for a more detailed presentation. We use here the same notations.

\subsection{Abstract syntax}

We consider here a kernel of \rml{}:
%
\begin{align*}
v \sdeq & c \mid (v, v) \mid n \mid \elam{x}{e} \mid \eproc{e} \\
e \sdeq &\phantom{\mid}  x \mid c \mid (e, e) \mid \elam{x}{e} 
  \mid \eapp{e}{e} \mid \erec{x}{e} \\
  & \mid \eproc{e} \mid \erun{e} \mid \epause{}  \\
& \mid \eletpar{x}{e}{x}{e}{e} \\
 &\mid \esig{x}{e}{e}{e} \\
 & \mid \eemit{e}{e} \mid \epres{e}{e}{e} \\
 & \mid \eite{e}{e}{e} \\
 &  \mid \eloop{e} \mid \euntil{e}{e}{x}{e} 
   \mid \ewhen{e}{e} 
\end{align*}

Values are constants (integers, booleans, etc.), pairs of values, signal names $n$, functions an processes. The language is a call-by-value lambda-calculus, extended with constructs for creating~($\mathtt{process}$) and running~($\mathtt{run}$) processes, waiting for the next instant~($\mathtt{pause}$), parallel definitions~($\mathtt{let/and}$), declaring signals~($\mathtt{signal}$), emitting a signal~($\mathtt{emit}$) and several control structures: the test of presence of a signal~($\mathtt{present}$), the unconditional loop~($\mathtt{loop}$), weak preemption~($\mathtt{do/until}$) and suspension~($\mathtt{do/when}$).  The expression $\euntil{e_1}{s}{x}{e_2}$ executes its body $e_1$ and, when signal~$s$ is present, stops the execution of~$e_1$ and the executes the continuation~$e_2$ on the next instant, binding~$x$ to the value of~$s$. We denote $\_$ variables that do not appear free in the body of a $\mathtt{let}$ and $\evoid{}$ the unique value of type $\mathtt{unit}$. From this kernel, we can encode most constructs of the language, as shown in Figure~\ref{fig:derived}.
%
\begin{figure*}

\begin{small}
\begin{align*}
%\eawaitim{e} \deq &\; \ewhen{\evoid{}}{e} \\
\epar{e_1}{e_2} \deq &\; \eletpar{\_}{e_1}{\_}{e_2}{\evoid{}} \\
\elet{x}{e_1}{e_2} \deq &\; \eletpar{x}{e_1}{\_}{\evoid{}}{e_2} \\
\mathidl{let} f\, x_1\, \ldots x_p = e_1 \mathid{in} e_2
      \deq &\; \elet{f}{\elam{x_1}{\ldots \elam{x_p}{e_1}}}{e_2} \\
\mathidl{let} \mathidl{rec} f\, x_1\, \ldots x_p = e_1 \mathid{in} e_2
      \deq &\; \elet{f}{ (\erec{f}{\elam{x_1}{\ldots \elam{x_p}{e_1}}}) }{e_2} \\
\mathidl{let} \mathidl{process} f\, x_1\, \ldots x_p = e_1 \mathid{in} e_2
      \deq &\; \elet{f}{\elam{x_1}{\ldots \elam{x_p}{ \eproc{e_1} }}}{e_2} \\     
\mathidl{let} \mathidl{rec} \mathidl{process} f\, x_1\, \ldots x_p = e_1 \mathid{in} e_2
            \deq &\; \elet{f}{ (\erec{f}{\elam{x_1}{\ldots \elam{x_p}{\eproc{e_1}}}}) }{e_2} \\
\eseq{e_1}{e_2} \deq &\; \elet{\_}{e_1}{e_2}  \\
\eawait{e_1}{x}{e_2} \deq&\; \euntil{ (\eloop{\epause{}}) }{e_1}{x}{e_2}
\end{align*}
\end{small}

\caption{Derived language constructs}
\label{fig:derived}
\end{figure*}


\subsection{Semantics}

A reaction of an expression is given by the relation
\[  \bigstep{N}{e}{e'}{E}{b}{S} \]
which means during the instant, in the signal environment $S$, the expression~$e$ rewrites to~$e'$ and emits the signals in $E$. $b$ is a boolean that indicates if $e'$ has terminated. We write $n \in S$ when the signal $n$ is present in the signal environment $S$, and $n \not\in S$ otherwise.

The execution of a program is made of a (potentially infinite) number of reactions and terminates when the status $b$ is equal to true. The execution of an instant is defined by the smallest signal environment $S_i$ such that:
\[ \bigstep{N}{e_i}{e_{i+1}}{E_i}{b_i}{S_i} \]
Additional conditions express for instance the fact that the emitted values in $E_i$ must agree with the signal environment $S_i$. Again we refer the reader to~\cite{Mandel:2005} for a more complete description.



Figure~\ref{fig:big_step} shows part of the rules defining the relation. The remaining rules can be found in Appendix~\ref{sec:big_step_other}. The rule for $\epause{}$ shows the meaning of the boolean $b$: if it is false, it means that the expression is stuck waiting for the next instant. The expression $\epres{e}{e_1}{e_2}$ executes its first branch immediately if the signal is present, but executes the second branch on the next instant if the signal is absent. This delayed reaction to absence, first introduced in~\cite{Boussinot:1991}, avoids any inconsistency in the presence of signals. The body of $\mathtt{do/when}$ is executed only during the instants when the signal is present. when this body reduces to a value, the $\mathtt{do/when}$ reduces immediately to the same value. The unconditional loop executes keeps executing its body until it awaits the next instant, that is, the termination status $b$ is equal to true. In particular, an expression like $\eloop{\evoid{}}$, where the body always terminates instantaneously, does not have a semantics as it would require an infinite derivation tree.

\begin{figure*}
\begin{small}
\begin{mathpar}
%
\inferrule
   { 
      \bigstep{N_1}{e_1}{\lambda x.\; e} {E_1}{\strue}{S} \\  
      \bigstep{N_2}{e_2}{v_2} {E_2}{\strue}{S} \\
      \bigstep{N_3}{e[x \leftarrow v_2]}{v} {E_3}{\strue}{S}
   }
   { \bigstep{N_1 \cdotp N_2 \cdotp N_3}{ \eapp{e_1}{e_2} }{v} {E_1 \sqcup E_2 \sqcup E_3}{\strue}{S} }
%
\and
%
\inferrule
   { \bigstep{N}{e[x \leftarrow \erec{x}{e}]}{v} {E}{\strue}{S} }
   { \bigstep{N}{\erec{x}{e}}{v} {E}{\strue}{S}  }
%
\and
%
\inferrule
  { \bigstep{N}{e} {\eproc{e_1}}  {E}{\strue}{S}  \\ 
    \bigstep{N_1}{e_1}{e'_1}  {E_1}{b}{S}   }
  { \bigstep{N \cdotp N_1}{\erun{e}}{e'_1} {E \sqcup E_1}{b}{S}  }
%
\and
%
\bigstep{\emptyset}{\epause{}}{\evoid{}}  {\emptyset}{\sfalse}{S}
%
\and
%
\inferrule
  { \bigstep{N}{e}{n} {E}{\strue}{S} \\ n \in S \\ 
    \bigstep{N_1}{e_1}{e'_1} {E_1}{b}{S} }
  { \bigstep{N \cdotp N_1}{\epres{e}{e_1}{e_2}}{e'_1} {E \sqcup  E_1}{b}{S}  }
%
\and
%
\inferrule
  { \bigstep{N}{e}{n} {E}{\strue}{S} \\ n \not\in S }
  { \bigstep{N}{\epres{e}{e_1}{e_2}}{e_2} {E}{\sfalse}{S} }
%
\\
%
\inferrule
  { \bigstep{N}{e}{n} {E}{\strue}{S} \\ n \not\in S }
  { \bigstep{N}{\ewhen{e_1}{e}}{\ewhen{e_1}{n}}  {E}{\sfalse}{S} }
%
\and
%
\inferrule
  { \bigstep{N}{e}{n} {E}{\strue}{S} \\ n \in S \\
     \bigstep{N_1}{e_1}{e'_1} {E_1}{\sfalse}{S}  }
  { \bigstep{N \cdotp N_1}{\ewhen{e_1}{e}}{\ewhen{e'_1}{n}}  {E \sqcup E_1}{\sfalse}{S} }
%
\and
%
\inferrule
  { \bigstep{N}{e}{n} {E}{\strue}{S} \\ n \in S \\
     \bigstep{N_1}{e_1}{v} {E_1}{\strue}{S}  }
  { \bigstep{N \cdotp N_1}{\ewhen{e_1}{e}}{v}  {E \sqcup E_1}{\strue}{S} }
%
\and
%
\inferrule 
  {  \bigstep{N}{e}{e'} {E}{\sfalse}{S}  }
  { \bigstep{N}{\eloop{e}}{\eseq{e'}{\eloop{e}}} {E}{\sfalse}{S}  }
%
\and
%
\inferrule 
  {  \bigstep{N_1}{e}{v} {E_1}{\strue}{S} \\ \bigstep{N_2}{\eloop{e}}{e'} {E_2}{b}{S}  }
  { \bigstep{N_1 \cdotp N_2}{\eloop{e}}{\eseq{e'}{\eloop{e}}} {E_1 \sqcup E_2}{\sfalse}{S}  }
%
\end{mathpar}

\end{small}


\caption{Big-step semantics}
\label{fig:big_step}
\end{figure*}

\section{The algebra of behaviors}

Behaviors are used to abstract the reactive behavior of processes. A main design choice is to completely abstract values and the presence of signals. It is however necessary to keep part of the structure of the program (or an abstraction of it) in order to have a precise analysis.

\subsection{The behaviors}

The algebra of behaviors is given by:
\[
\rk \sdeq \effplus ~\mid~ \effmin ~\mid~ \effvar ~\mid~ \effpar{\rk}{\rk} ~\mid~ \effor{\rk}{\rk}
~\mid~ \effcon{\rk}{\rk}  ~\mid~ \effrec{\effvar}{\rk}  ~\mid~ \effrun{\rk}
\]

It is first necessary to distinguish (surely) instantaneous actions that takes at least one instant to execute, such as $\epause{}$, from (potentially) non-instantaneous ones like calling a pure ML function or emitting a signal. The former are denoted $\effplus{}$ while the latter are denoted $\effmin$.

Behaviors must reflect the structure of the program, starting with parallel composition and alternatives, as shown by the following examples: 
\begin{lstlisting}
let process par_comb q1 q2 =
  loop
    run q1 || run q2
  end

let process if_comb c q1 q2 =
  loop
    if c then run q1 else run q2
  end
\end{lstlisting}
In the case of the `par_comb` process, the loop is non-instantaneous if either `q1` or `q2` is non-instantaneous, while in the case of `if_comb`, both processes must be non-instantaneous. That is why behaviors include the parallel composition, simply denoted $||$, and a non-deterministic choice operator denoted $+$. This shows how values are abstracted: we only keep the different alternatives and forget about the conditions.

It is also necessary to have a notion of sequence, denoted $;$ in the language of behaviors, as illustrated by the following example. The order between the recursive call and the call to `pause` is crucial as the `good_rec` process is correct while `bad_rec` loops instantaneously:
\begin{lstlisting}
let rec process good_rec =
  pause; run good_rec

let rec process bad_rec =
    run bad_rec; pause
\end{lstlisting}
As it is defined recursively, the behavior  $\rk$  associated to `good_rec` must verify that~$ \rk = \effcon{\effplus}{\effrun{\rk}}$. The $\mathtt{run}$ operator is associated to running a process and is necessary to solve technical problems in the type system, that will be discussed in Section~\ref{sec:type_system}. This equation can be solved by introducing an explicit recursion operator $\mu$ so that $\rk = \effrec{\effvar}{\effplus}{\effrun{\effvar}}$. Similarly, the behavior of an unconditional loop is given by $\effinf{\rk}$ where $\rk$ is the behavior of the body of the loop. It intuitively represents the infinite repetition of the behavior $\rk$ and can be defined as a recursive behavior by $\effinf{\rk} \deq \effrec{\effvar}{\effcon{\rk}{\effrun{\effvar}}}$. 

\begin{figure}[t]

\subfloat[Reactive behaviors]{
\begin{small}
\begin{mathpar}
%
\inferrule{ }{ \wf{R}{\effmin} }
%
\and
%
\inferrule{ }{ \wf{R}{\effplus} }
%
\and
%
\inferrule
  { \effvar \not\in R }
  { \wf{R}{\effvar} }
%
\and
%
\inferrule
  { \wf{R}{\rk_1} \\ \noinst{\rk_1} \\ \wf{\emptyset}{\rk_2} }
  { \wf{R}{\effcon{\rk_1}{\rk_2}} }
%
\and
%
\inferrule
  { \wf{R}{\rk_1} \\ \mathit{not}(\noinst{\rk_1})   \\ \wf{R}{\rk_2} }
  { \wf{R}{\effcon{\rk_1}{\rk_2}} }
%
\and
%
\inferrule
  { \wf{R}{\rk_1} \\ \wf{R}{\rk_2} }
  { \wf{R}{\effpar{\rk_1}{\rk_2}} }
%
\and
%
\inferrule
  { \wf{R}{\rk_1} \\ \wf{R}{\rk_2} }
  { \wf{R}{\effor{\rk_1}{\rk_2}} }
%
\and
%
\inferrule
  { \wf{R \cup \{ \effvar \}}{\rk} } %\effvar \in fbv(\rk) \\
  { \wf{R}{\effrec{\effvar}{\rk}} }
%
\and
%
\inferrule
  {\wf{R}{\rk}}
  {\wf{R}{\effrun{\rk}}}
\end{mathpar}

\end{small}
\label{fig:reactive_behavior}
}

\subfloat[Non-instantaneous behaviors]{
\begin{small}
\begin{mathpar}
\inferrule{ }{ \noinst{\effplus} }
%
\and
%
\inferrule{ }{ \noinst{\effvar} }
%
\and
%
\inferrule
  { \noinst{\rk_1} }
  {\noinst{\effcon{\rk_1}{\rk_2}}} 
%
\and
%
\inferrule
  { \noinst{\rk_2} }
  {\noinst{\effcon{\rk_1}{\rk_2}}} 
%
\and
%
\inferrule
  { \noinst{\rk_1} }
  {\noinst{\effpar{\rk_1}{\rk_2}}} 
%
\and
%
\inferrule
  { \noinst{\rk_2} }
  {\noinst{\effpar{\rk_1}{\rk_2}}} 
%
\\
%
\inferrule
  { \noinst{\rk_1} \\ \noinst{\rk_2} }
  {\noinst{\effor{\rk_1}{\rk_2}}} 
%
\and
%
%\inferrule
%  {\effvar \in fbv(\rk) \lor \noinst{\rk}}
%  {\noinst{\effrec{\effvar}{\rk}}}
%\noinst{\effrec{\effvar}{\rk}}
\inferrule
  { \noinst{\rk} }
  { \noinst{\effrec{\effvar}{\rk}}  }
%
\and
%
\inferrule
  {\noinst{\rk}}
  {\noinst{\effrun{\rk}}}
\end{mathpar}
\end{small}
\label{fig:noinst_behavior}
}

\caption{Properties of behaviors}
\end{figure}


\subsection{Reactive behaviors}

Using the language of behaviors, we can now characterize the behaviors that we want to reject, that is instantaneous loops and recursions. We actually enforce a stronger sufficient condition, that can be checked easily and efficiently: there must be at least one instant before each recursive call. This condition is not necessary, as the `par_map` example of Section~\ref{} showed. As the behavior of a loop is just a particular case of recursive behavior, this condition also implies that the body of a loop is non-instantaneous.

A behavior that respect this condition is called a \emph{reactive} behavior. A behavior is reactive if for each recursive behavior $\effrec{\effvar}{\rk}$, the recursion variable $\effvar$ does not appear in the first instant of the body $\rk$. Formally, we define a predicate $\wf{R}{\rk}$ meaning that the behavior $\rk$ is reactive with respect to the set of variables $R$, that is, these variables do not appear in the first instant of $\rk$ and all the recursions inside $\rk$ are well-formed. This relation is defined in Figure~\ref{fig:reactive_behavior}. It uses the notion of \emph{non-instantaneous} behavior, denoted $\noinst{\rk}$ and defined in Figure~\ref{fig:noinst_behavior}. Indeed, in the case of $\eseq{\rk_1}{\rk_2}$, if the behavior $\rk_1$ is non-instantaneous, then it is not necessary to check if the variables in $R$ appear free in $\rk_2$. However, we still have to check that $\rk_2$ is reactive, that is, that $\wf{[]}{\rk_2}$.

\section{The type-and-effect system}
\label{sec:type_system}

\subsection{Notations}

\begin{align*}
\rt \sdeq &\;\;\; \rtvar \mid T \mid \rtprod{\rt}{\rt} \mid \rtarrow{\rt}{\rt} \\
    &\;\mid \rtproc{\rt}{\rk} \mid \rtsig{\rt}{\rt} && \text{(types)} \\
\rs \sdeq &\; \rt \mid \forall \effvar.\, \rs \mid \forall \rtvar.\, \rs && \text{(type schemes)} \\
\tyenv \sdeq&\; \emptyset \mid \tyenv \tyconcat x : \rs && \text{(environments)}
\end{align*}

We denote $ftv(\rt)$ (resp. $fbv(\rt)$) the set of type~(resp. beavior) variables free in $\rt$ and $fv(\rt) = ftv(\rt), fbv(\rt)$. Instantiation and generalization are defined in a classic way:
\begin{mathpar}
\rs[\rtvar \leftarrow \rt] \leq \forall \rtvar.\, \rs 
\and
\rs[\effvar \leftarrow \rk] \leq \forall \effvar.\, \rs  
\end{mathpar}
\vspace{-1.3em}
\begin{align*}
gen(\rt, e, \tyenv) = &\, \rt && \text { if $e$ is expansive} \\
gen(\rt, e, \tyenv) = &\; \forall \bar{\rtvar}.\forall \bar{\effvar}.\; \rt 
   \text{ where }  \bar{\rtvar}, \bar{\effvar} = fv(\rt) \setminus fv(\tyenv)  
   && \text{ otherwise } 
\end{align*}

The initial typing environment is: 
\begin{align*}
\Gamma_0 \deq [
 &\eemit{}{} : \forall  \rtvar_1, \rtvar_2. \, 
        \rtarrow{ \rtsig{\rtvar_1}{\rtvar_2} }{\rtarrow{\rtvar_1}{\unit{}} }; \\
        &\mathtt{true} : \mathtt{bool};
              \mathtt{fst} : \forall \rtvar_1, \rtvar_2.\, \rtarrow{\rtprod{\rtvar_1}{\rtvar_2}}{\rtvar_1};  \ldots ]
\end{align*}

\subsection{Typing rules}

\begin{figure*}
\begin{small}
\begin{mathpar}
%\inferrule[Equiv]
%  { \tyju{\tyenv}{e}{\rt}{\rk_1} \\ \effeq{\rk_1}{\rk_2} }
%  { \tyju{\tyenv}{e}{\rt}{\rk_2} }
%%
%\and
%
\inferrule%[Var]
  { \rt \leq \tyenv(x) }
  { \tyju{\tyenv}{x}{\rt}{\effmin} }
%
\and
%
\inferrule%[Const]
  { \rt \leq \tyenv_0(c) }
  { \tyju{\tyenv}{c}{\rt}{\effmin} }
%
\and
%
\inferrule%[Pair]
  { \tyju{\tyenv}{e_1}{\rt_1}{\effmin} \\ 
  	\tyju{\tyenv}{e_1}{\rt_2}{\effmin}  }
  { \tyju{\tyenv}{(e_1, e_2)}{ \rtprod{\rt_1}{\rt_2} }{\effmin} }
%
\and
%
\inferrule%[Abs]
  { \tyju{\tyenv \tyconcat x : \rt_1} {e} {\rt_2} {\effmin} } 
  { \tyju{\tyenv} { \elam{x}{e} }   { \rtarrow{\rt_1}{\rt_2} } {\effmin} }
%
\and
%
\inferrule%[App]
  {  \tyju{\tyenv} {e_1}{ \rtarrow{\rt_2}{\rt_1} } {\effmin} \\
  	 \tyju{\tyenv} {e_2}{\rt_2} {\effmin}  }
  { \tyju{\tyenv} {\eapp{e_1}{e_2}} {\rt_1} {\effmin} }
%
\and
%
\inferrule%[Rec]
  { \tyju{\tyenv \tyconcat x : \rt}{e}{\rt}{\effmin} }
  {  \tyju{\tyenv}{ \erec{x}{e} }{\rt}{\effmin}  }
%
%\and
%
%\inferrule%[Seq]
%  { \tyju{\tyenv}{e_1}{\rt_1}{\rk_1}  \\  \tyju{\tyenv}{e_2}{\rt_2}{\rk_2} }
%  { \tyju{\tyenv}{ \eseq{e_1}{e_2} } {\rt_2} {\effcon{\rk_1}{\rk_2}} }
%
\and
%
\rulename{Process}
\inferrule%[ProcAbs]
  { \tyju{\tyenv}{e}{\rt}{\rk}   }
  {  \tyju{\tyenv}{ \eproc{e} }{\rtproc{\rt}{\effor{\rk}{\rk'}}}{\effmin} }
%
\and
%
\inferrule%[ProcApp]
  {  \tyju{\tyenv}{e}{\rtproc{\rt}{\rk}}{\effmin} }
  {  \tyju{\tyenv}{ \erun{e} }{\rt}{\effrun{\rk}} }
%
\and
%
\tyju{\tyenv}{\epause{}}{\mathtt{unit}}{\effplus} 
%
\and
%
\inferrule%[If]
  { \tyju{\tyenv}{e_1}{\mathtt{bool}}{\effmin} \\
    \tyju{\tyenv}{e_2}{\rt}{\rk_2} \\
    \tyju{\tyenv}{e_3}{\rt}{\rk_3} }
  { \tyju{\tyenv} {\eite{e_1}{e_2}{e_3}} {\rt}{\effor{\rk_2}{\rk_3}} }
%
\and
%
\inferrule%[Par]
  { \tyju{\tyenv}{e_1}{\rt_1}{\rk_1}  \\  
     \tyju{\tyenv}{e_2}{\rt_2}{\rk_2} \\
     \tyju{\tyenv \tyconcat  x_1 : gen(\rt_1, e_1, \tyenv) 
                         \tyconcat x_2 : gen(\rt_2, e_2, \tyenv)}
             {e}{\rt}{\rk} \\
    }
  {  \tyju{\tyenv}{ \eletpar{x_1}{e_1}{x_2}{e_2}{e} } {\rt} { \effcon{ (\effpar{\rk_1}{\rk_2}) }{\rk} } }
%
\and
%
\inferrule%*[Left=(Signal)]
  { \tyju{\tyenv}{e_1}{\rt_2}{\effmin}  \\ 
    \tyju{\tyenv}{e_2}{ \rtarrow{\rt_1}{\rtarrow{\rt_2}{\rt_2}} }{\effmin} \\\\
    \tyju{\tyenv \tyconcat x : \rtsig{\rt_1}{\rt_2}} {e} {\rt}{\rk}  }
  { \tyju{\tyenv}{ \esig{x}{e_1}{e_2}{e} } {\rt}{ \effcon{\effmin}{\rk}} }
%
\and
%
\inferrule%[Present]
  { \tyju{\tyenv}{e_1}{\rtsig{\rt_1}{\rt_2}}{\effmin} \\\\
    \tyju{\tyenv}{e_2}{\rt}{\rk_2}  \\
    \tyju{\tyenv}{e_3}{\rt}{\rk_3}  }
  { \tyju{\tyenv} { \epres{e_1}{e_2}{e_3} }{\rt}{ \effor{\rk_2}{(\effcon{\effplus}{\rk_3})} } }
%
\and
%
%\inferrule%[Emit]
%  { \tyju{\tyenv}{e_1}{\rtsig{\rt_1}{\rt_2}}{\effmin} \\ 
%    \tyju{\tyenv}{e_2}{\rt_1}{\effmin} }
%  { \tyju{\tyenv}{\eemit{e_1}{e_2}}{\mathtt{unit}}{\effmin} }
%%
%\and
%
\inferrule%[Loop]
  { \tyju{\tyenv}{e}{\rt}{\rk} }
  { \tyju{\tyenv}{\eloop{e}}{\unit}{\effinf{\rk}} }
%
\and
%
\inferrule%[Until]
  { \tyju{\tyenv}{e_1}{\rt}{\rk_1} \\ 
    \tyju{\tyenv}{e_2}{\rtsig{\rt_1}{\rt_2}}{\effmin} \\\\
    \tyju{\tyenv \tyconcat x : \rt_2}{e_3}{\rt}{\rk_3}  }
  { \tyju{\tyenv}{ \euntil{e_1}{e_2}{x}{e_3} } {\rt} { \effor{\rk_1}{\effcon{\effplus}{\rk_3}}  } } 
%
\\
%
\inferrule%[When]
  { \tyju{\tyenv}{e_1}{\rt}{\rk} \\\\
    \tyju{\tyenv}{e_2}{\rtsig{\rt_1}{\rt_2}}{\effmin} }
  { \tyju{\tyenv}{ \ewhen{e_1}{e_2} } {\rt} { \effor{\rk}{\effinf{\effplus}} } } 
%
\and
%
\rulename{Mask}
\inferrule
  { \tyju{\tyenv}{e}{\rt}{\rk} \\ \effvar \not\in fbv(\tyenv, \rt) }
  { \tyju{\tyenv}{e}{\rt}{\rk[\effvar \leftarrow \effplus]  } }
\end{mathpar}

\end{small}

\caption{Type-and-effect rules}
\label{fig:rules}
\end{figure*}

The rules are given in Figure~\ref{fig:rules}.

\begin{itemize}

\item On ne cherche pas à prouver la terminaison des fonctions \ocaml{} pures (sans comportement réactif): la règle de l'application montre que l'on suppose que les appels de fonctions terminent toujours. C'est pourquoi il n'y a pas de comportement associé à une fonction, puisqu'il est toujours instantané (il n'y a pas de comportement sur les flèches contrairement à la tradition dans les systèmes de types et effets).

\item Dans le modèle synchrone réactif, on ne peut pas réagir instantanément à l'absence d'un signal. C'est pourquoi dans le cas du $\mathtt{present}$ la branche $\mathtt{then}$ est exécutée immédiatement si le signal est présent et la branche $\mathtt{else}$ à l'instant suivant dans le cas contraire. Cela se traduit de façon immédiate dans le comportement de $\mathtt{present}$.
  %
  De même, dans un $\mathtt{until}$, en cas de préemption, la
  continuation est exécutée à l'instant suivant la réception du
  signal. Cela est reflété dans la règle de typage.
% De la même façon, la préemption dans le cas du $\mathtt{until}$ est faible, puisque la continuation est exécutée à l'instant suivant, ce que l'on voit clairement sur la règle de typage.

%\item Le comportement obtenu pour $\mathtt{present}$ peut sembler un peu étrange. En suivant au plus près la sémantique, on choisirait un comportement $\effor{\rk_2}{(\effcon{\effplus}{\rk_3})}$: on exécute la première branche instantanément ou la seconde à l'instant suivant. On peut montrer simplement que ce comportement est bien équivalent à $\rk_2$:
%\[
%\inferrule
%  {
%    \inferrule
%      {
%        \effeq{\effcon{\effplus}{\rk_3}}{\rk_3}\\
%        \effeq{\rk_2}{\rk_2}
%      }
%      { 
%      \effeq{\effor{\rk_2}{(\effcon{\effplus}{\rk_3})}}{\effor{\rk_2}{\effplus}} 
%      }
%      \\  
%      \effeq{\effor{\rk_2}{\effplus}}{\rk_2}
%  }
%  { \effeq{\effor{\rk_2}{(\effcon{\effplus}{\rk_3})}}{\rk_2} }
%\]
%Le même raisonnement permet de justifier le comportement attribué à $\mathtt{until}$.

\pagebreak
\item On peut vérifier que l'encodage des primitives que l'on a donné permet bien de retrouver les comportements attendus. Par exemple, on peut considérer le cas de $\eseq{e_1}{e_2}$:
\begin{small}
\[ 
\inferrule
{
\inferrule%*[left=(Equiv)]
  { 
    \inferrule 
      {
        \tyju{\tyenv}{e_1}{\rt_1}{\rk_1} \\
        \tyju{\tyenv}{\evoid}{\unit}{\effmin} \\
        \tyju{\tyenv}{e_2}{\rt_2}{\rk_2} \\
      }
      { 
      \tyju{\tyenv}{\eletpar{\_}{e_1}{\_}{\evoid}{e_2}}{\rt_2}{ \effcon{(\effpar{\rk_1}{\effmin})}{\rk_2} } 
      } 
  }
  { \tyju{\tyenv}{\eseq{e_1}{e_2}}{\rt}{ \effcon{(\effpar{\rk_1}{\effmin})}{\rk_2} } } 
 \\
 \inferrule
   { \effeq{\effpar{\rk_1}{\effmin}}{\rk_1}  \\ \effeq{\rk_2}{\rk_2} }
   { \effeq{ \effcon{(\effpar{\rk_1}{\effmin})}{\rk_2} } { \effcon{\rk_1}{\rk_2} } }
}
{
  \tyju{\tyenv}{\eseq{e_1}{e_2}}{\rt_2}{ \effcon{\rk_1}{\rk_2} }
}
\]
\end{small}
On peut de la même façon vérifier que $\epar{e_1}{e_2}$ a bien un comportement équivalent à $\effpar{\rk_1}{\rk_2}$ ou que $\eawait{e_1}{x}{e_2}$ a le comportement $\effeq{\effor{\effinf{\effplus}}{(\effcon{\effplus}{\rk_2})}}{\effcon{\effplus}{\rk_2}}$ après simplification. % ou encore que $\eawaitim{e_1}$ a un comportement instantané $\effmin$. 

\item Il n'y a pas de règles particulières pour les processus récursifs. C'est l'ajout de l'opérateur $\mathtt{run}$ qui force le comportement des processus récursifs à être de la forme $\effrec{\effvar}{\rk}$ avec $\effvar \in fbv(\rk)$.
Considérons par exemple le processus défini par \lstinline+let rec process p = run p+, que l'on écrit $\erec{p}{\eproc{(\erun{p})}}$ dans notre noyau: 
\begin{small}
\[
\inferrule
{
\inferrule
  {
   \inferrule*[left=Equiv]
     {
     \inferrule*
      { \tyju{\tyenv \tyconcat p : \rtproc{\beta}{\rk}}{p}{\rtproc{\beta}{\rk}}{\effmin} }
      { \tyju{\tyenv \tyconcat p : \rtproc{\beta}{\rk}}{\erun{p}} {\beta}{\effrun{\rk}} }
      \\
      \effeq{\effrun{\rk}}{\rk}
     }
     {
     \tyju{\tyenv \tyconcat p : \rtproc{\beta}{\rk}}{\erun{p}} {\beta}{\rk}
     }
%        \qquad \boxed{\rtvar = \rtproc{\beta}{\rk}}  
  }  
  { \tyju{\tyenv \tyconcat p : \rtproc{\beta}{\rk}}{\eproc{(\erun{p})}} {\rtproc{\beta}{\rk}}{\effmin} }
}
{ \tyju{\tyenv}{ \erec{p}{\eproc{(\erun{p})}} }{\rtproc{\beta}{\rk}}{\effmin} }
%\qquad \boxed{\rtvar = \rtproc{\beta}{\rk}}  
\]
\end{small}
On voit sur la dérivation de typage que l'on doit vérifier $\effeq{\effrun{\rk}}{\rk}$ pour pouvoir typer cette expression. Seul le comportement $\effrec{\effvar}{\effrun{\effvar}}$ vérifie cette propriété. Cela explique aussi pourquoi aucune règle d'équivalence ne concerne l'opérateur \texttt{run}, qui ne doit pas être simplifiable. En effet, sans le $\mathtt{run}$, la condition $\effeq{\rk}{\rk}$ serait toujours vraie. On aurait donc pu prendre par exemple $\rk = \effmin$ et rater la récursion instantanée.

{
\abovedisplayskip=0.5em
\belowdisplayskip=0.5em
\item La construction $\mathtt{loop}$ ne fait pas habituellement partie du noyau du langage puisque l'on peut l'obtenir par:
\[
\eloop{e} \deq \mathid{run}(( \mathidl{rec} loop = 
     \elam{x}{\eproc{(\eseq{\erun{x}}{\erun{(\eapp{loop}{x})}})}})\; (\eproc{e})) 
\]
Si l'on applique les règles du système de types et en supposant que \mbox{$\tyju{\tyenv}{e}{\rt}{\rk}$}, on obtient les types suivants:
\begin{align*}
loop:&\, \forall \effvar.\, \rtarrow{\rtproc{\rtvar}{\effvar}}
                {\rtproc{\rtvar'}{\effrec{\effvar'}{\effcon{\effrun{\effvar}}{\effrun{\effvar'}}}}}  \\
\eproc{e} :&\, \rtproc{\rt}{\rk}
\end{align*}
On obtient donc le comportement $\effrun{(\effrec{\effvar'}{\effcon{\effrun{\rk}}{\effrun{\effvar'}}})}$ pour $\eloop{e}$. Ce comportement n'est pas équivalent à $\effinf{\rk}$, mais il est réactif si et seulement si $\effinf{\rk}$ est réactif, puisque les $\mathtt{run}$ n'influent pas sur la réactivité des comportements. On aurait donc pu enlever $\mathtt{loop}$ du noyau présenté ici sans aucune incidence sur le résultat de l'analyse de réactivité.

\item Ce n'est par contre pas le cas pour $\epause{}$, que l'on peut encoder par:
\[
\epause{} \deq \esig{s}{\evoid}{(\elam{x}{\elam{y}{\mathtt{\evoid}}})}{\epres{s}{\evoid}{\evoid}}
\]
Nous avons fait le choix de complètement abstraire les valeurs et les présences des signaux. Comme dans l'exemple \texttt{imprecise} de la partie~\ref{sec:intuition}, on ne prend pas en compte le fait que le signal~$s$ n'est jamais présent et que seule la seconde branche du $\mathtt{present}$ est exécutée. On obtient donc le comportement $\effeq{\effor{\effmin}{\effcon{\effplus}{\effmin}}}{\effmin}$ qui n'est pas du tout ce que l'on souhaite.
}
\end{itemize}

\subsection{Subeffecting with row polymorphism}

\begin{itemize}
\item Remind subeffecting with constraints: enough to have a conservative extension of the type system (see \cite{Nielson:1999} for instance). Subeffecting = replacing an effect with a bigger effect. Here, replace $\rk$ with $\effor{\rk}{\rk'}$.
\[
\inferrule
  { \tyju{\tyenv}{e}{\rt}{\rk} \\ \rk \sqsubseteq \rk' }
  { \tyju{\tyenv}{e}{\rt}{\rk'}  }
\]

In \cite{Amtoft:1999}, add a constraint set to the typing rules, simple types, ie effects are always variables and:
\[
\inferrule
  { \tyju{\tyenv, C}{e}{\rt}{\rk}   }
  {  \tyju{\tyenv, C \cup {\rk \sqsubseteq \effvar} }{ \eproc{e} }{\rtproc{\rt}{\effvar}}{\effmin} }
\]
For sets (like regions), $\sqsubseteq$ is subset inclusion. In our case, such that $\rk \sqsubseteq \effor{\rk}{\rk'}$.
\item Behaviors of processes are always variables or rows (we could have enforced it using different kinds)
\item Show extended unification algorithm:no explicit constraints, only collection of effects and unification
\end{itemize}

%L'algorithme prend en entrée en environnement de typage $\tyenv$ et une expression $e$ et renvoie une substitution $S$, un type $\rt$ et un effet $\rk$:
%\[ \mW(\tyenv, e) = S, \rt, \rk \]

%Description de l'algorithme:
%\begin{small}
%\begin{align*}
%\mW(\tyenv, \evoid) =&\,(id, \mathtt{unit}, \effmin) \\
%\mW(\tyenv, x) =&\, (id, \rt[\bar{\alpha} \leftarrow \bar{\beta}], \effmin) 
%          \text{ où } \tyenv(x) = \forall \bar{\alpha}.\, \rt &&  \fresh{\bar{\beta}} \\
%\mW(\tyenv, \elam{x}{e}) =&\, \mlet (S_1, \rt_1, \rk_1) = \mW(\tyenv \tyconcat x:\beta, e) && \fresh{\beta} \\
%                                          &\,  \mIn (S_1, S_1 \rtarrow{\beta}{\effor{\rk_1}{\effvar} }{\rt_1}, \effmin) \\
%\mW(\tyenv, \eapp{e_1}{e_2}) =&\, \mlet (S_1, \rt_1, \rk_1) = \mW(\tyenv, e_1)  \\
%												   &\, \mand S'_1 = \mU_{\rk}(\rk_1, \effmin) \\
%												   &\, \mand (S_2, \rt_2, \rk_2) = \mW(S'_1 S_1\tyenv, e_2) \\
%												   &\, \mand S'_2 = \mU_{\rk}(\rk_2, \effmin) \\
%												   &\, \mand S_3 = \mU(S'_2 S_2 \rt_1, \rt_2 \rightarrow \beta) && \fresh{\beta} \\
%												   &\, \mIn (S_3 S'_2 S_2 S'_1 S_1, S_3\beta) \\
%\mW(\tyenv, \elet{x}{e_1}{e_2}) =&\, \mlet (S_1, \rt_1, \rk_1) = \mW(\tyenv, e_1)  \\
%													 &\, \mand (S_2, \rt_2, \rk_2)= \mW(S_1\tyenv \tyconcat x : gen(\rt, e_1, \tyenv), e_2) \\
%													 &\, \mIn (S_2 S_1, \rt_2, \rk_1 \cup \rk_2) \\
%\mW(\tyenv, \erec{x}{e}) =&\, \mlet (S_1, \rt_1) = \mW(\tyenv \tyconcat x:\beta, e) && \fresh{\beta} \\
%										  &\, \mand  S_2 = \mU(S_1\beta, \rt_1) \\
%										  &\, \mIn (S_2 S_1, S_2 \rt_1)
%\end{align*}
%\end{small}

The inference algorithm is the algorithm $\mW$ with the collection of behaviors. Can be easily implemented in any ML type inference algorithm.

L'algorithme d'unification prend en entrée deux types $\rt_1$ et $\rt_2$ et renvoie une substitution $S$ ou échoue avec la valeur spéciale $\fail$:
\[  \mU(\rt_1, \rt_2) = S \]

Improved unification algorithm of 
\begin{small}
\begin{align*}
\mU_{\rk}(\effvar, \rk) = \mU_{\rk}(\rk, \effvar) =
					&\, [\effvar \mapsto  \effrec{\effvar'}{\rk[\effvar \leftarrow \effvar']} ]  \text{ if } \mathtt{occur\_check}(\effvar, \rt) \\
					&\,  [\effvar \mapsto \rk] \\
\mU_{\rk}(\effor{\rk_1}{\effvar_1}, \effor{\rk_2}{\effvar_2}) =&\,
                  [\effvar_1 \mapsto \effor{\rk_2}{\effvar}; \\
                  &\,\; \effvar_1 \mapsto \effor{\rk_2}{\effvar} ] && \fresh{\effvar} \\
\mU_{\rk}(\rk, \rk) =&\, [] \\
\mU_{\rk}(\rk_1, \rk_2) =&\, \fail{} && \text{ dans tous les autres cas}
\end{align*}
\end{small}


\subsection{Simplifying behaviors}

\begin{itemize}
\item In the proof of correction, we only use the fact that the behavior is reactive.
\item We introduce an equivalence on behaviors, that preserve reactivity
\item Add a new typing rule (some kind of subtyping), can also simplify some rules
\end{itemize}

Elle est définie de telle sorte que si $\effeq{\rk_1}{\rk_2}$ et $\wf{R}{\rk_1}$ alors $\wf{R}{\rk_2}$.

La relation $\effeq{}{}$ est une relation d'équivalence, c'est-à-dire une relation réflexive, symétrique et transitive. Les opérateurs $\effcon{}{}$ et $\effpar{}{}$ et $\effor{}{}$ sont compatibles avec cette relation d'équivalence et idempotents. 
$\effcon{}{}$ et $\effpar{}{}$ et $\effor{}{}$ sont associatifs. $\effpar{}{}$~et $\effor{}{}$ sont commutatifs (mais pas $;\,$). Le comportement~$\effmin$~(resp.~$\effplus$) est l'élément neutre de $\effcon{}{}$ et $\effpar{}{}$ (resp.~$\effor{}{}$). On ajoute aussi les règles suivantes, qui expriment les propriétés de l'opérateur de récursion et traduisent l'intuition de récursion:
%
\begin{mathpar}
\inferrule
  { \effeq{\rk_1}{\rk_2} }
  { \effeq{\effrec{\effvar}{\rk_1}}{\effrec{\effvar}{\rk_2}} }
%
\and
%
\effeq{\effrec{\effvar}{\rk}}{\rk[\effvar \leftarrow \effrec{\effvar}{\rk}]}
%
\and
%
\effeq{\effinf{\effplus}}{\effplus}
\end{mathpar}

On peut par exemple montrer que le comportement
$\effrec{\effvar}{(\effcon{(\effpar{\effplus}{\effmin})}{(\effor{\effrun{\effvar}}{\effrun{\effvar}})})}$
est équivalent à
$\effrec{\effvar}{\effcon{\effplus}{\effrun{\effvar}}}$.

New typing rule:
\[ 
\inferrule
  { \tyju{\tyenv}{e}{\rt}{\rk_1} \\ \effeq{\rk_1}{\rk_2} }
  { \tyju{\tyenv}{e}{\rt}{\rk_2} } 
\]

\subsection{Examples}

\section{Proof of soundness}


On définit une opération $\rkfst{\rk}$ qui ne garde que l'effet correspondant au premier instant:
\begin{align*}
\rkfst{\effmin} =&\, \rkfst{\effplus}  = \effmin \\
\rkfst{\effvar} =&\, \effvar \\
\rkfst{\effrun{\rk}} =&\, \effrun{(\rkfst{\rk})} \\ 
\rkfst{\effpar{\rk_1}{\rk_2}} =&\, \effpar{\rkfst{\rk_1}}{\rkfst{\rk2}} \\
\rkfst{\effor{\rk_1}{\rk_2}} =&\, \effor{\rkfst{\rk_1}}{\rkfst{\rk2}} \\
\rkfst{\effcon{\rk_1}{\rk_2}} =&\, \effcon{\rk_1}{\rkfst{\rk2}} &&\text{ si } \rkfst{\rk_1} = \rk_1\\
\rkfst{\effcon{\rk_1}{\rk_2}} =&\, \rkfst{\rk_1} && \text{ sinon } \\
\rkfst{\effrec{\effvar}{\rk}} =&\, \rkfst{\rk[\effvar \leftarrow \effrec{\effvar}{\rk}]}
\end{align*}
Dans le cas d'un comportement récursif, $\rkfst{\rk}$ est bien défini si le comportement est réactif, mais peut ne pas etre défini sinon.

\begin{property}
Si le comportement $\rk$ est réactif, alors $\rkfst{\rk}$ est un comportement fini (i.e. il est équivalent à un comportement sans comportement récursif).
\end{property}
Cette propriété est la base de la preuve de correction. Le comportement de tout processus bien typé est fini, donc on sait que l'on va appeler un nombre fini de processus au cours de l'instant. On montre ainsi que l'exécution de l'instant est finie, sous réserve que les fonctions terminent.

\begin{theorem}[Sureté]
Si $\tyju{\tyenv}{e}{\rt}{\rk}$ et si on suppose que "les fonctions terminent", alors il existe $e'$ tel que $\bigstep{N}{e}{e'} {E}{b}{S}$ et $\tyju{\tyenv}{e'}{\rt}{\rk'}$.
\end{theorem}
\begin{proof}
On va montrer par récurrence sur la taille de $\rkfst{\rk}$ que l'on peut obtenir une dérivation finie dans la sémantique à grands pas de \rml{} pour tout terme bien typé. L'hypothèse que "les fonctions terminent" se traduit plus formellement par le fait qu'on suppose qu'il existe une dérivation finie $\Pi$ pour toute expression $e$ (surement) instantanée, c'est-à-dire:
\[ 0 \vdash e \Rightarrow \exists v.\, \inferrule{\Pi}{ \bigstep{N}{e}{v} {E}{b}{S} } \] 
(Il y a la preuve dans le Lemme 8 p. 81 de la thèse de Louis que les expressions instantanées se réduisent instantanément en une valeur ou se réduisent infiniment. On suppose ici que le deuxième cas ne peut pas arriver.)

Dans le cas des primitives réactives, on utilise l'hypothèse de récurrence sur les sous-expressions, qui ont un comportement plus petit:
\begin{itemize}
\item Cas de $\erun{e}$: On a alors que $0 \vdash e$ donc il existe $\Pi$ tel que 
\[ \inferrule{\Pi}{ \bigstep{N_1}{e}{\eproc{e_1}} {E}{true}{S} } \]
De plus, comme $\erun{e}$ est bien typé, on a:
\[
\inferrule
  {
  \inferrule
    { \tyju{\tyenv}{e_1} {\rt}{\rk}  }
    { \tyju{\tyenv}{\eproc{e_1}} { \rtproc{\rt}{\rk} }{\effmin} }
  }
  { \tyju{\tyenv}{\erun{(\eproc{e_1})}} {\rt}{\effrun{\rk}} }
\]
On peut appliquer l'hypothèse de récurrence sur $e_1$ et on obtient donc que:
\[  \inferrule{\Pi_1}{ \bigstep{N_2}{e_1}{e'_1} {E_1}{b}{S} } \]
et on peut donc construire la dérivation complète pour $\erun{e}$:
\[
\inferrule
{
\inferrule{\Pi}{ \bigstep{N_1}{e}{\eproc{e_1}} {E}{true}{S} } \\
\inferrule{\Pi_1}{ \bigstep{N_2}{e_1}{e'_1} {E_1}{b}{S} }
}
{ \bigstep{N_1 \cdotp N_2}{\erun{e}}{e'_1} {E \sqcup E_1}{b}{S} }
\]

\item Cas de $\epause{}$: La dérivation est finie sans aucune hypothèse.

\item Cas de $\epres{e}{e_1}{e_2}$: 
De la meme façon, on prouve que :
\[ \inferrule{\Pi}{ \bigstep{N_1}{e}{n} {E}{true}{S} } \]
La règle de typage nous donne:
\[ 
\inferrule%[Present]
  { \tyju{\tyenv}{e}{\rtsig{\rt_1}{\rt_2}}{\effmin} \\
    \tyju{\tyenv}{e_1}{\rt}{\rk_1}  \\
    \tyju{\tyenv}{e_2}{\rt}{\rk_2}  }
  { \tyju{\tyenv} { \epres{e}{e_1}{e_2} }{\rt}{ \effor{\rk_1}{(\effcon{\effplus}{\rk_2})} } }
\]
On doit traiter les deux cas, selon la présence du signal:
\begin{itemize}
\item Cas où $n \in S$: Alors on applique l'hypothèse de récurrence sur $e_1$  et on conclut simplement.
\item Cas où $n \not\in S$: Alors la dérivation est finie.
\end{itemize}

\item Cas de $\eletpar{x_1}{e_1}{x_2}{e_2}{e}$: Idem.

\item Cas de $\euntil{e_1}{e_2}{x}{e_3}$: Idem.

\item Cas de $\esig{x}{e_1}{e_2}{e}$: On utilise l'hypothèse sur les expressions instantanées pour $e_1$ et $e_2$. Pour traiter le cas de $e$, on doit supposer que la règle de typage est la suivante:
\[
\inferrule%*[Left=(Signal)]
  { \tyju{\tyenv}{e_1}{\rt_2}{\effmin}  \\ 
    \tyju{\tyenv}{e_2}{ \rtarrow{\rt_1}{\rtarrow{\rt_2}{\rt_2}} }{\effmin} \\
    \tyju{\tyenv \tyconcat x : \rtsig{\rt_1}{\rt_2}} {e} {\rt}{\rk}  }
  { \tyju{\tyenv}{ \esig{x}{e_1}{e_2}{e} } {\rt}{ \effcon{\effmin}{\rk}} }
\]
Le comportement donné à l'expression est équivalent mais permet d'avoir un comportement qui décroit.

\item Cas de $\ewhen{e_1}{e_2}$: On modifie de la meme façon la règle de typage devient:
\[
\inferrule%[When]
  { \tyju{\tyenv}{e_1}{\rt}{\rk} \\
    \tyju{\tyenv}{e_2}{\rtsig{\rt_1}{\rt_2}}{\effmin} }
  { \tyju{\tyenv}{ \ewhen{e_1}{e_2} } {\rt} { \effor{\rk}{\effinf{\effplus}} } } 
\]

\end{itemize}

Pour prouver que l'on obtient une expression bien typée pour l'instant suivant, on étend la preuve de correction du typage, qui est faite sur la sémantique à petits pas (dont l'équivalence avec la sémantique à grands pas a été prouvée), pour prendre en compte les comportements. On n'a besoin d'aucune propriété sur le comportement $\rk'$ de $e'$, puisqu'on n'utilise seulement le fait que ce comportement est réactif (qui est inclut dans la définition de bien typé). 

\end{proof}

\section{Related work}


\section{Conclusion}

\bibliographystyle{plain}

\begin{small}
\bibliography{biblio}
\end{small}


\appendix

\section{Other rules for the big-step semantics}
\label{sec:big_step_other}

\todo{Mettre autres regles}

\end{document}
